\documentclass{article}

\usepackage{caption}
\usepackage{courier}
\title{Example Tables}
\author{Calvin Whealton}

\begin{document}
\maketitle

\listoftables

\newpage

This is an example file showing how regression tables from R can be imported into \LaTeX using the code available on \texttt{https://github.com/calvinwhealton/R\_tables\_to\_LaTeX}.

\section{Ordinary Least Squares}

\begin{table}[h!]
\caption{A table where only the table values are written, not the caption, label, and horizontal lines at the beginning and end of the table.}\label{ols1}
\begin{center}
\begin{tabular}{l l l l l}
\hline
%	2016-02-03 14:42:42	
% output from R regression
SSE	&	92.82	&		&	DoF	&	97\\ 
SSR	&	3422.78	&		&	R2	&	0.9736\\ 
SST	&	3515.6	&		&	Adj.R2	&	0.9731\\ 
RSE	&	0.978	&		&		&	\\ 
\hline
Variable	&	Est. Coeff.	&	Std. Error	&	t Stat.	&	Pr($>|t|$)\\ 
\hline
(Intercept)	&	4.787e+00	&	2.619e-01	&	1.827e+01	&	3.554e-33\\ 
x1	&	5.136e-01	&	3.791e-02	&	1.355e+01	&	4.231e-24\\ 
x2	&	2.011e+00	&	3.445e-02	&	5.839e+01	&	2.232e-77\\ 

\hline
\end{tabular}
\end{center}
\end{table}

%	2016-02-03 14:42:42	
% output from R regression
% This is an example for writing an entire table from R 
\begin{table}[h!]
\caption[OLS table all from R]{A table that is entirely written from R including captions (long and short) and label. The parameter names were defined in as different from the model output as well.}\label{ols2}\begin{center}
\begin{tabular}{l l l l l}

\hline
SSE	&	92.82	&		&	DoF	&	97\\ 
SSR	&	3422.78	&		&	R2	&	0.9736\\ 
SST	&	3515.6	&		&	Adj.R2	&	0.9731\\ 
RSE	&	0.978	&		&		&	\\ 
\hline
Variable	&	Est. Coeff.	&	Std. Error	&	t Stat.	&	Pr($>|t|$)\\ 
\hline
b0	&	4.787e+00	&	2.619e-01	&	1.827e+01	&	3.554e-33\\ 
bx1	&	5.136e-01	&	3.791e-02	&	1.355e+01	&	4.231e-24\\ 
bx2	&	2.011e+00	&	3.445e-02	&	5.839e+01	&	2.232e-77\\ 
\hline\end{tabular}
\end{center}
\end{table}

%	2016-02-03 15:16:46	
% output from R regression
% Regression does not include a constant, R^2 = 1 - SSE/SST
% This is an example for writing an entire table from R 
\begin{table}[h!]
\caption[OLS table all from R, no Constant]{A table that is entirely written from R including captions (long and short) and label. There is no constant in this model.}\label{ols3}\begin{center}
\begin{tabular}{l l l l l}

\hline
SSE	&	324.67	&		&	DoF	&	98\\ 
SSR	&	3708.83	&		&	R2	&	0.8595\\ 
SST	&	2310.82	&		&	Adj.R2	&	0.8566\\ 
RSE	&	1.82	&		&		&	\\ 
\hline
Variable	&	Est. Coeff.	&	Std. Error	&	t Stat.	&	Pr($>|t|$)\\ 
\hline
cx1	&	1.018e+00	&	9.588e-02	&	1.062e+01	&	5.537e-18\\ 
cx2	&	2.496e+00	&	7.688e-02	&	3.246e+01	&	3.078e-54\\ 
\hline\end{tabular}
\end{center}
\end{table}

\section{Nonlinear Least Squares}

\begin{table}[h!]
\caption{A nonlinear least squares regression table with the center components written from R.}\label{nls1}
\begin{center}
\begin{tabular}{l l l l l}
\hline
%	2016-02-03 15:53:24
% output from R non-linear least squares regression
SSE	&	94.26	&		&	DoF	&	97\\ 
SSR	&	122.68	&		&	R2\_star	&	0.5654\\ 
SST	&	216.91	&		&		&	\\ 
RSE	&	0.986	&		&		&	\\ 
\hline
Variable	&	Est. Coeff.	&	Std. Error	&	t Stat.	&	Pr($>|t|$)\\ 
\hline
c0	&	5.089e+00	&	2.983e-01	&	1.706e+01	&	5.837e-31\\ 
c1	&	1.071e-01	&	2.579e-02	&	4.153e+00	&	7.06e-05\\ 
c2	&	1.851e+00	&	1.989e-01	&	9.304e+00	&	4.28e-15\\ 

\hline
\end{tabular}
\end{center}
\end{table}

%	2016-02-03 15:54:32
% output from R non-linear least squares regression
% This is an example for writing an entire table from R including mathematical notation 
\begin{table}[h!]
\caption[NLS table all from R]{Nonlinear regression table written from R. Model is $y = \mu + \exp(\alpha x_1) + \beta \ln(x_2)$. The caption includes mathematical notation.}\label{nls2}\begin{center}
\begin{tabular}{l l l l l}

\hline
SSE	&	94.26	&		&	DoF	&	97\\ 
SSR	&	122.68	&		&	R2\_star	&	0.5654\\ 
SST	&	216.91	&		&		&	\\ 
RSE	&	0.986	&		&		&	\\ 
\hline
Variable	&	Est. Coeff.	&	Std. Error	&	t Stat.	&	Pr($>|t|$)\\ 
\hline
$\mu$	&	5.089e+00	&	2.983e-01	&	1.706e+01	&	5.837e-31\\ 
$\alpha$	&	1.071e-01	&	2.579e-02	&	4.153e+00	&	7.06e-05\\ 
$\beta$	&	1.851e+00	&	1.989e-01	&	9.304e+00	&	4.28e-15\\ 
\hline\end{tabular}
\end{center}
\end{table}

\section{Quantile Regression}

\begin{table}[h!]
\caption{A Quantile regression table from R for a linear model. AE is for the absolute error and the number represents the percentile.}\label{qreg1}
\begin{center}
\begin{tabular}{l l l l l}
\hline
%	2016-02-04 15:40:01
% output from R quantile regression
% Quantile of interest is 0.5
AE 0.10	&	0.175	&		&	DoF	&	100\\ 
AE 0.25	&	0.425	&		&	MAE	&	0.827\\ 
AE 0.50	&	0.813	&		&	pseudoR2	&	0.9833\\ 
AE 0.75	&	1.221	&		&	Reps.	&	5000\\ 
AE 0.90	&	1.521	&		&		&	\\ 
\hline
Variable	&	Est. Coeff.	&	Std. Error	&	t Stat.	&	Pr($>|t|$)\\ 
\hline
(Intercept)	&	5.113e+00	&	5.871e-01	&	8.709e+00	&	3.067e-18\\ 
x1	&	0e+00	&	0e+00	&	0e+00	&	0e+00\\ 
x2	&	0e+00	&	0e+00	&	0e+00	&	0e+00\\ 

\hline
\end{tabular}
\end{center}
\end{table}

%	2016-02-04 15:40:33
% output from R quantile regression
% Quantile of interest is 0.5
% This is an example for writing an entire table from R 
\begin{table}[h!]
\caption[Quantile regression table all from R]{A quantile regression table from R.}\label{qreg}\begin{center}
\begin{tabular}{l l l l l}

\hline
AE 0.10	&	0.175	&		&	DoF	&	97\\ 
AE 0.25	&	0.425	&		&	MAE	&	0.827\\ 
AE 0.50	&	0.813	&		&	pseudoR2	&	0.9833\\ 
AE 0.75	&	1.221	&		&	Reps.	&	5000\\ 
AE 0.90	&	1.521	&		&		&	\\ 
\hline
Variable	&	Est. Coeff.	&	Std. Error	&	t Stat.	&	Pr($>|t|$)\\ 
\hline
b0	&	5.113e+00	&	5.871e-01	&	8.709e+00	&	3.067e-18\\ 
bx1	&	4.958e-01	&	6.786e-03	&	7.307e+01	&	0e+00\\ 
bx2	&	2.002e+00	&	6.164e-03	&	3.247e+02	&	0e+00\\ 
\hline\end{tabular}
\end{center}
\end{table}

\section{Data Frames}

\begin{table}[h!]
\caption{A data drame written from R but with other table information (caption, label, environment) written in \LaTeX.}\label{df1}
\begin{center}
\begin{tabular}{l r c}
\hline
%	2016-02-05 11:12:29
% R data frame
y	&	x1	&	x2\\ 
\hline
1.6e+02	&	5.1e+01	&	6.5e+01\\ 
1.3e+02	&	3.1e+01	&	5.7e+01\\ 
4.9e+01	&	4.3e+01	&	1.1e+01\\ 
1.6e+02	&	6.9e+01	&	6e+01\\ 
8.2e+01	&	8.5e+00	&	3.6e+01\\ 
1e+02	&	2.3e+01	&	4.3e+01\\ 
2.8e+01	&	2.7e+01	&	5.2e+00\\ 
7.1e+01	&	2.7e+01	&	2.6e+01\\ 
1.2e+02	&	6.2e+01	&	4e+01\\ 
1.9e+02	&	4.3e+01	&	8.4e+01\\ 

\hline
\end{tabular}
\end{center}
\end{table}

%	2016-02-05 11:12:29
% R data frame
\begin{table}[h!]
\caption{A data frame with different numbers of significant figures. There is no short caption.}\label{df2}\begin{center}
\begin{tabular}{l l l }

\hline
A	&	B	&	C\\ 
\hline
1.6e+02	&	5.075e+01	&	6.51656e+01\\ 
1.3e+02	&	3.068e+01	&	5.67738e+01\\ 
4.9e+01	&	4.269e+01	&	1.13509e+01\\ 
1.6e+02	&	6.931e+01	&	5.95925e+01\\ 
8.2e+01	&	8.514e+00	&	3.5805e+01\\ 
1e+02	&	2.254e+01	&	4.28809e+01\\ 
2.8e+01	&	2.745e+01	&	5.19033e+00\\ 
7.1e+01	&	2.723e+01	&	2.64178e+01\\ 
1.2e+02	&	6.158e+01	&	3.98791e+01\\ 
1.9e+02	&	4.297e+01	&	8.36134e+01\\ 
\hline\end{tabular}
\end{center}
\end{table}


\end{document}